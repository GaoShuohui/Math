\documentclass{article}
% 这里是导言区
\begin{document}
hello, world!
$\alpha$
x

\begin{equation}
M\cdot\frac{d^{2}s}{dt^{2}}+\frac{nF}{l}\cdot s = \frac{nFh}{l}-Mg
\label{2}
\end{equation}

\begin{equation}
s=k_1\cos\left (  \sqrt{ \frac{nF}{lm}} t\right ) +k_2\sin\left ( \sqrt{\frac{nF}{lm}} t\right ) +h-\frac{Mgl}{nF}  
\end{equation}


\begin{equation}
s=\left ( \frac{Mgl}{nF}-h \right )\cos\left (  \sqrt{ \frac{nF}{lm}} t\right ) +h-\frac{Mgl}{nF}  
\end{equation}


\begin{equation}
\sqrt{\frac{nF}{lM}}\left ( \frac{Mgl}{nF}-h \right )=\frac{m\sqrt{2g\left ( h_0+h-\frac{Mgl}{nF} \right )}}{M}
\end{equation}

\begin{equation}
mv-Mu=-m{v}'+M{u}'
\end{equation}

\begin{equation}
v+u={u}'+{v}'
\end{equation}

\begin{equation}
u=\frac{mv}{M}
\end{equation}	
\\
2.5 A

$A=M\left ( \frac{R^2}{2}+\frac{d^2}{3} \right ) \\$
$C=2\sum\limits_{k=1}^{\frac{n}{2}-1}F_{k,0}R\cos\frac{2\pi k}{n}\cos\varphi 
	+ F_{0,0} R\sin\varphi
	- F_{\frac{n}{2},0} R\sin\varphi
 \\ $
$D=-2\sum\limits_{k=1}^{\frac{n}{2}-1}F_{k,0} R\cos^2\frac{2\pi k}{n}\sin\varphi
	-F_{0,0} R\cos\varphi
	-F_{\frac{n}{2},0} R\cos\varphi 
\\$
$A\cdot \frac{d^{2}\psi }{dt^{2} }=C\cos\psi+D\sin \psi
\\$
\\
2.5 B

$A=M\left ( \frac{R^2}{2}+\frac{d^2}{3} \right ) \\ $
$C= 2\sum\limits_{k=1}^{\frac{n}{2}}F_{k,0}R\cos\left ( \frac{2\pi k}{n}-\frac{\pi}{n} \right)\sin\varphi
\\ $
$D=-2\sum\limits_{k=1}^{\frac{n}{2}}F_{k,0} R\cos^2\left ( \frac{2\pi k}{n}-\frac{\pi}{n} \right )\cos\varphi
\\ $
$A\cdot \frac{d^{2}\psi }{dt^{2} }=C\cos\psi+D\sin \psi
\\$
\\
2.5 C

$A=M\left ( \frac{R^2}{2}+\frac{d^2}{3} \right ) \\$
$C=2\sum\limits_{k=1}^{\frac{n}{2}-1}F_{k,0}R\cos\frac{2\pi k}{n}\sin\varphi 
	+ F_{0,0} R\sin\varphi
 \\ $
$D=-2\sum\limits_{k=1}^{\frac{n}{2}-1}F_{k,0} R\cos^2\frac{2\pi k}{n}\cos\varphi
	-F_{0,0} R\cos\varphi
\\$
$A\cdot \frac{d^{2}\psi }{dt^{2} }=C\cos\psi+D\sin \psi
\\$


2.8

%M是力矩

$M= \left ( 
	\sum\limits_{k=1}^{n}F_{k,0} R \sin\varphi \sin \left ( \frac{2\pi k}{n}-\frac{2\pi}{n} \right ) ,
	\sum\limits_{k=1}^{n}F_{k,0} R \sin\varphi \cos \left ( \frac{2\pi k}{n}-\frac{2\pi}{n} \right ) ,
	0 \right )
\\ $
%从而
$\psi_0 = \frac{M_0}{2J}\cdot \Delta t^2
\\$
$\omega_0  = \frac{M_0}{J}\cdot \Delta t
\\$
%0-0.1s时间内
$
\vec{\psi} = \left ( 
	\vec{\omega_0}\cdot \Delta t + \vec{\frac{M}{2J}}\cdot \Delta t^2
	\right ) \bigoplus \psi_0 
\\$
% \bigoplus:旋转的复合

%令
$M  = 2 \Delta F_0 R \sin\varphi \cos \frac{3\pi}{8}
\\$
$G  = \omega_0 \Delta t
\\$
$H  = \frac{M}{2J}\cdot \Delta t^2
\\$

%旋转轴:

$\left ( 
	G \cos \frac{3\pi}{8}+H \cos \frac{5\pi}{8},
	G \sin \frac{3\pi}{8}+H \sin \frac{5\pi}{8},0
\right )
\\$
%设转轴为
$ \left ( K,L,0 \right )
\\$
%其中
$ \Delta \psi  = \sqrt{G^2+H^2+\sqrt2 GH}
\\$
$K  =  \frac {\left ( G-H \right ) \cos \frac{3\pi}{8} }{\left | \Delta\psi \right | }
\\$
$L  =  \frac {\left ( G+H \right ) \sin \frac{3\pi}{8} }{\left | \Delta\psi \right | }
\\$

%用a表示Δpsi所代表的四元数
%用b表示psi_0所代表的四元数

$ a = \cos\frac {\Delta \psi}{2} 
		+K \sin\frac {\Delta \psi}{2} \cdot i 
		+L \sin\frac {\Delta \psi}{2} \cdot j
\\$
$ b = \cos\frac {\varphi_0}{2} 
		+\cos \frac{3\pi}{8} \sin\frac {\varphi_0}{2} \cdot i 
		+\sin \frac{3\pi}{8} \sin\frac {\varphi_0}{2} \cdot j
\\$

%ψ的表达式:(对应的四元数)

$ \psi=a\circ b \\$
$		= \cos\frac{\Delta \psi}{2} \cos\frac{\varphi_0}{2} 
		 -\sin\frac{\Delta \psi}{2} \sin\frac{\varphi_0}{2} 
			\left (
				\cos \frac{3\pi}{8} \cdot K - \sin \frac{3\pi}{8}\cdot L
			\right ) \\$

$		+ \left (
				\sin\frac{\Delta \psi}{2} \cos\frac{\varphi_0}{2} \cdot K 
			+  \sin\frac{\Delta \psi}{2} \cos\frac{\varphi_0}{2} \cos\frac{3\pi}{8} 
\right )	\cdot i	\\$

$		+ \left (
				\sin\frac{\Delta \psi}{2} \cos\frac{\varphi_0}{2} \cdot L
			+  \sin\frac{\Delta \psi}{2} \cos\frac{\varphi_0}{2} \sin\frac{3\pi}{8} 
\right )	\cdot j	\\$

$		+ \left [
				\sin\frac{\Delta \psi}{2} \sin\frac{\varphi_0}{2} 
				\left (
					\sin \frac{3\pi}{8} \cdot K - \cos \frac{3\pi}{8}\cdot L
				\right )
			\right ]	\cdot k	\\$
%记为
$ \psi=P+iQ+jS+kU \\$
%其中(大括号)
$	 P = \cos\frac{\Delta \psi}{2} \cos\frac{\varphi_0}{2} 
		 -\sin\frac{\Delta \psi}{2} \sin\frac{\varphi_0}{2} 
			\left (
				\cos \frac{3\pi}{8} \cdot K - \sin \frac{3\pi}{8}\cdot L
			\right ) \\$

$		Q =
				\sin\frac{\Delta \psi}{2} \cos\frac{\varphi_0}{2} \cdot K 
			+  \sin\frac{\Delta \psi}{2} \cos\frac{\varphi_0}{2} \cos\frac{3\pi}{8} 
	\\$

$		S =
				\sin\frac{\Delta \psi}{2} \cos\frac{\varphi_0}{2} \cdot L
			+  \sin\frac{\Delta \psi}{2} \cos\frac{\varphi_0}{2} \sin\frac{3\pi}{8} 
	\\$

$		U =
				\sin\frac{\Delta \psi}{2} \sin\frac{\varphi_0}{2} 
				\left (
					\sin \frac{3\pi}{8} \cdot K - \cos \frac{3\pi}{8}\cdot L
				\right )
	\\$

%所以ψ(的大小)等于
$ \psi=2\arccos{P} \\$
%旋转轴上一点的坐标为
$ (Q,S,U) \\$
%通过求解计算,得出PQSU等于……,所以我们可以近似地认为旋转轴为y轴(0,S,0)
%cos的值为arccosP
\end{document}