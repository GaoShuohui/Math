% !Mode:: "TeX:UTF-8"
% !TEX program  = xelatex

%\documentclass{cumcmthesis}
\documentclass[withoutpreface,bwprint]{cumcmthesis} %去掉封面与编号页
\usepackage[framemethod=TikZ]{mdframed}
\usepackage{url}   % 网页链接
\usepackage{subcaption} % 子标题
\title{多人协作同心鼓的优化模型}
\tihao{A}
\baominghao{4321}
\schoolname{XX大学}
\membera{小米}
\memberb{向左}
\memberc{哈哈}
\supervisor{老师}
\yearinput{2017}
\monthinput{08}
\dayinput{22}

\begin{document}

 \maketitle
 \begin{abstract}






\keywords{\TeX{}\quad  图片\quad   表格\quad  公式}
\end{abstract}


\section{问题的重述}
\subsection{问题背景}

同心鼓是一项团队运动,用以训练团队协作的默契,参加的人需要拉住与鼓相连的绳子末端,使得球在单位时间内在鼓上连续弹跳的次数尽可能的多,因此,球的弹跳高度应该尽量低,即鼓与球相撞时,鼓的速度应尽可能小,而鼓面也应当保持水平,以避免球做斜抛运动而掉出鼓面,这要求队员需一起精准地控制在绳上的用力大小与用力时机。
同时,该运动有以下规则限制:
\begin{enumerate}
    \item 球离鼓面的弹跳高度不能低于40cm
    \item 参加的人数不少于8人且间距应小于60cm
    \item 球初始落下的位置为球面上方40cm
    \item 同心鼓与球有标准的大小与重量要求
\end{enumerate}


\subsection{本文需求解的问题}
\begin{enumerate}
    \item 在可精确控制情况下,建立模型计算出参加该运动的人数、绳长、用力时机、用力大小的最佳值,以使单位时间内球在鼓面上连续弹跳次数最多。
    \item 建立队员用力大小、用力时机与鼓面倾角之间的关系的模型,根据所给的数据,计算出相应的鼓面倾角。
    \item 考虑问题二中出现的鼓面倾斜情况,相应地调整问题一的策略。
    \item 在可精确控制情况下,根据具体给出的鼓面倾斜数据,计算出要使球恢复竖直弹跳,队员应当用力的时机与大小。
\end{enumerate}

\section{模型的假设}

\begin{itemize}
    \item 鼓面与球面光滑,鼓与球之间没有摩擦。
    \item 每个人的持绳高度相同。
    \item 绳子没有弹性
	\item 队员站位不变且间距都相等
	\item 碰撞过程时间不计
	\item 空气阻力不计
	\item 鼓与球间为完全弹性碰撞
	\item 球初始位置参照的鼓面高度为绳子水平时的鼓面高度
\end{itemize}



\section{变量的说明}

\begin{table}[!htbp]
    \caption{标准三线表格}\label{tab:001} \centering
    \begin{tabular}{cccc}
        \toprule[1.5pt]
        变量 & 说明 & 变量 &  说明 \\
        \midrule[1pt]
        5 & 269.8 & 0.000674 & 1.79  \\
        10 & 421.0 & 0.001035 & 3.59 \\
        20 & 640.2 & 0.001565 & 7.18 \\
        \bottomrule[1.5pt]
    \end{tabular}
\end{table}




\section{模型的建立与求解}

%问题一
\subsection{问题一}
	\subsubsection{问题分析与建模思路}
对于问题一,要使得单位时间内球弹跳次数尽量多,则应使得球与鼓碰撞后的速度尽可能地小,即使球的弹跳高度尽可能低,而在球的弹跳高度有最低限制的情况下,显然球的弹跳高度应当恰为限制的最低高度,可达到最优结果。
但同时由于题目没有限定人数与绳长,在有球弹跳的最优高度的情况下,人数与绳长的组合可以有无限种,而在实际情况下,人数与绳长都是有限的,不可能达到无限多或无限长,因此为了评价不同的人数与绳长组合的方案,本文以单人用力大小和用地面积为评价标准,认为单人用力大小与用地面积越小时,该方案越佳。

	\subsubsection{模型的建立}
为了使每个人的用力都尽可能小,本文首先设定所有人的用力大小、时机、方向相同,鼓在向上运动的速度最大时与球相撞,队员按照周期用力,
同时,人数不可能无限增加,因此只讨论n小于等于20时的情况;人的手臂长度也是有限的,在拉绳时绳末端与人之间的距离不能改变过大,因此鼓的初始位置离水平面不能过低,本文限定绳长与深度之比满足:。本文设计的评价指标为:方案等级=单人的用力大小绳长2,当方案等级越小时,认为该方案越佳。
对鼓进行受力分析,如图所示:
因为绳与水平面的夹角极小,可近似认为人与鼓间距离即为绳长,列出方程如下:
球下落时与鼓相撞的时间方程如下:
球与鼓之间为完全弹性碰撞,根据动量守恒可列出方程如下:
人的间距不能小于60cm,则可列出以下方程:
联立后则可列出二次方程如下:

	\subsubsection{模型的求解}

%问题二

\subsection{问题二}
	\subsubsection{问题分析与建模思路}


	\subsubsection{模型的建立}


	\subsubsection{模型的求解}

%问题三
\subsection{问题三}
	\subsubsection{问题分析与建模思路}


	\subsubsection{模型的建立}


	\subsubsection{模型的求解}

%模型的评价与推广
\section{模型的评价与推广}


%参考文献
\section{参考文献}

\begin{thebibliography}{9}%宽度9
    \bibitem[1]{liuhaiyang2013latex}
    刘海洋.
    \newblock \LaTeX {}入门\allowbreak[J].
    \newblock 电子工业出版社, 北京, 2013.
    \bibitem[2]{mathematical-modeling}
    全国大学生数学建模竞赛论文格式规范 (2019 年 9 月 12 日修改).
\end{thebibliography}







\end{document} 

%无序列表是这样的:
%\begin{itemize}
%    \item one
%    \item two
%    \item ...
%\end{itemize}
%
%有序列表是这样子的:
%\begin{enumerate}
%    \item one
%    \item two
%    \item ...
%\end{enumerate}



%下面简单介绍一下定理、证明等环境的使用。
%\begin{definition}
%    定义环境
%    \label{def:nosense}
%\end{definition}
%\cref{def:nosense}除了告诉你怎么使用这个环境以外,没有什么其它的意义。
%
%除了 definition 环境,还可以使用 theorem 、lemma、corollary、assumption、conjecture、axiom、principle、problem、example、proof、solution 这些环境,根据论文的实际需求合理使用。
%
%\begin{theorem}
%    这是一个定理。
%    \label{thm:example}
%\end{theorem}
%由\cref{thm:example}我们知道了定理环境的使用。
%
%\begin{lemma}
%    这是一个引理。
%    \label{lem:example}
%\end{lemma}
%由\cref{lem:example}我们知道了引理环境的使用。
%
%\begin{corollary}
%    这是一个推论。
%    \label{cor:example}
%\end{corollary}
%由\cref{cor:example}我们知道了推论环境的使用。
%
%\begin{assumption}
%    这是一个假设。
%    \label{asu:example}
%\end{assumption}
%由\cref{asu:example}我们知道了假设环境的使用。
%
%\begin{conjecture}
%    这是一个猜想。
%    \label{con:example}
%\end{conjecture}
%由\cref{con:example}我们知道了猜想环境的使用。
%
%\begin{axiom}
%    这是一个公理。
%    \label{axi:example}
%\end{axiom}
%由\cref{axi:example}我们知道了公理环境的使用。
%
%\begin{principle}
%    这是一个定律。
%    \label{pri:example}
%\end{principle}
%由\cref{pri:example}我们知道了定律环境的使用。
%
%\begin{problem}
%    这是一个问题。
%    \label{pro:example}
%\end{problem}
%由\cref{pro:example}我们知道了问题环境的使用。
%
%\begin{example}
%    这是一个例子。
%    \label{exa:example}
%\end{example}
%由\cref{exa:example}我们知道了例子环境的使用。
%
%\begin{proof}
%    这是一个证明。
%    \label{prf:example}
%\end{proof}
%由\cref{prf:example}我们知道了证明环境的使用。
%
%\begin{solution}
%    这是一个解。
%    \label{sol:example}
%\end{solution}
%由\cref{sol:example}我们知道了解环境的使用。
%


%
%\subsection{字体加粗与斜体}
%
%如果想强调部分内容,可以使用加粗的手段来实现。加粗字体可以用 \verb|\textbf{加粗}| 来实现。例如: \textbf{这是加粗的字体。 This is bold fonts} 。
%
%中文字体没有斜体设计,但是英文字体有。\textit{斜体 Italics}。
