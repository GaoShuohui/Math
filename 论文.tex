% !Mode:: "TeX:UTF-8"
% !TEX program  = xelatex

%\documentclass{cumcmthesis}
\documentclass[withoutpreface,bwprint]{cumcmthesis} %去掉封面与编号页
\usepackage[framemethod=TikZ]{mdframed}
\usepackage{url}   % 网页链接
\usepackage{subcaption} % 子标题
\title{多人协作同心鼓的优化模型}
\tihao{A}
\baominghao{4321}
\schoolname{XX大学}
\membera{小米}
\memberb{向左}
\memberc{哈哈}
\supervisor{老师}
\yearinput{2017}
\monthinput{08}
\dayinput{22}

\begin{document}

 \maketitle
 \begin{abstract}






\keywords{\TeX{}\quad  图片\quad   表格\quad  公式}
\end{abstract}


\section{问题的重述}
\subsection{问题背景}

同心鼓是一项团队运动,用以训练团队协作的默契,参加的人需要拉住与鼓相连的绳子末端,使得球在单位时间内在鼓上连续弹跳的次数尽可能的多,因此,球的弹跳高度应该尽量低,即鼓与球相撞时,鼓的速度应尽可能小,而鼓面也应当保持水平,以避免球做斜抛运动而掉出鼓面,这要求队员需一起精准地控制在绳上的用力大小与用力时机。
同时,该运动有以下规则限制:
\begin{enumerate}
    \item 球离鼓面的弹跳高度不能低于40cm
    \item 参加的人数不少于8人且间距应小于60cm
    \item 球初始落下的位置为球面上方40cm
    \item 同心鼓与球有标准的大小与重量要求
\end{enumerate}


\subsection{本文需求解的问题}
\begin{enumerate}
    \item 在可精确控制情况下,建立模型计算出参加该运动的人数、绳长、用力时机、用力大小的最佳值,以使单位时间内球在鼓面上连续弹跳次数最多。
    \item 建立队员用力大小、用力时机与鼓面倾角之间的关系的模型,根据所给的数据,计算出相应的鼓面倾角。
    \item 考虑问题二中出现的鼓面倾斜情况,相应地调整问题一的策略。
    \item 在可精确控制情况下,根据具体给出的鼓面倾斜数据,计算出要使球恢复竖直弹跳,队员应当用力的时机与大小。
\end{enumerate}

\section{模型的假设}

\begin{itemize}
    \item 鼓面与球面光滑,鼓与球之间没有摩擦。
    \item 每个人的持绳高度相同。
    \item 绳子没有弹性。
	\item 队员站位不变且间距都相等。
	\item 碰撞过程时间不计。
	\item 空气阻力不计。
	\item 鼓与球间为完全弹性碰撞。
	\item 球初始位置参照的鼓面高度为绳子水平时的鼓面高度。
\end{itemize}



\section{变量的说明}

\begin{table}[!htbp]
    \caption{变量说明}\label{tab:001} \centering
    \begin{tabular}{cccc}
        \toprule[1.5pt]
        变量 & 说明 & 变量 &  说明 \\
        \midrule[1pt]
        $m$ & 球的质量 & $t$ & 鼓开始运动的时间点  \\
        $M$ & 同心鼓质量 & $T$ & 队员发力的周期 \\
        $F_0$ & 鼓静止时的单人所需拉力 & $n$ & 队员人数 \\
         $F$ & 鼓运动时的单人所需拉力 & $R$ & 鼓的半径 \\
         $l$ & 绳长 & $d$ & 鼓高度的一半 \\
         $h$  & 鼓静止时离绳水平时所在平面的距离 & $l_0$ & 相邻两队员间距 \\
          $v$ & 相撞前一瞬间球的速率 & $\rho$ & 鼓的面密度  \\
          $v'$ & 相撞后一瞬间球的速率 & $\varphi$ & 绳与水平面夹角  \\
         $u $& 相撞前一瞬间鼓的速率 & $\psi$ & 鼓面与水平面夹角 \\
         $u'$ & 相撞后一瞬间鼓的速率 & $F_k$ & 编号为k的队员拉力大小  \\

        \bottomrule[1.5pt]
    \end{tabular}
\end{table}




\section{模型的建立与求解}

%问题一
\subsection{问题一}
	\subsubsection{问题分析与建模思路}
对于问题一,要使得单位时间内球弹跳次数尽量多,则应使得球与鼓碰撞后的速度尽可能地小,即使球的弹跳高度尽可能低,而在球的弹跳高度有最低限制的情况下,显然球的弹跳高度应当恰为限制的最低高度,可达到最优结果。
\par
但同时由于题目没有限定人数与绳长,在有球弹跳的最优高度的情况下,人数与绳长的组合可以有无限种,而在实际情况下,人数与绳长都是有限的,不可能达到无限多或无限长,因此为了评价不同的人数与绳长组合的方案,本文以单人用力大小和用地面积为评价标准,认为单人用力大小与用地面积越小时,该方案越佳。

	\subsubsection{模型的建立}
为了使每个人的用力都尽可能小并简化问题,本文首先设定所有人的用力大小、时机、方向相同,鼓与球相撞时速度恰达到最大,鼓的加速度为0,设球从水平面以上40cm的位置开始下落的时间点为原点,队员按照周期用力,球与鼓的运动分析如下:在时间段$\left [ 0,t \right ]$内,人对绳拉力为$F_0$,鼓静止,球自由落体;在时间段$\left [ t,T-t \right ] $内,人对绳拉力为F,鼓先做加速运动,球自由落体,在$\frac{T}{2}$时与球完全弹性碰撞,由动量守恒可知,

\begin{equation}
\begin{cases}


mv-Mu=-m{v}'+M{u}'    \\
	v+u={u}'+{v}' 

\end{cases}
\end{equation}


可得 $u=-{u}'$, $v=-{v}'$ ,即碰撞后球与鼓速度方向反转,速度大小不变,鼓做减速运动,球做上抛运动;在时间段$ \left [ T-t,T \right ]$内,人对绳拉力为$F_0$,鼓静止,球做上抛运动直至最高点,周期为$T$。可列运动方程如下:

\begin{equation}
\begin{cases}
     M\cdot\frac{d^{2}s}{dt^{2}}+\frac{nF}{l}\cdot s = \frac{nFh}{l}-Mg \\
	    s\left ( 0\right )=0 \\
       s'\left ( 0 \right )=0 \\
s'(\frac{T}{2}-t)=0
\end{cases}
\end{equation}
\par
解得,
\begin{equation}
\begin{cases}
    t=-\frac{\pi}{2}\sqrt{\frac{lM}{nF}}+\frac{v}{g}\\
	T=\frac{2v}{g}
\end{cases}
\end{equation}


\par
同时,人数不可能无限增加,因此只讨论$n\leq 20$时的情况;人的手臂长度也是有限的,在拉绳时绳末端与人之间的距离不能改变过大,因此鼓的初始位置离水平面不能过低,本文限定深度与绳长之比满足$h:l\leq 1:10$。使用方案等级$Z$作为评价指标,方案等级越小,说明单人用力大小与占地面积综合看越小,则认为该方案越佳。
\begin{equation}
Z=Fl^2
\end{equation}

\par
对鼓进行受力分析,如图所示:
因为绳与水平面的夹角极小,可近似认为人与鼓间距离即为绳长,列出方程如下:
\begin{align}
\sum F  &= n F\cdot \frac{h-s}{l}-M g \\
 &=M\cdot \frac{d^{2}s}{dt^{2}}
\label{1}
\end{align}


联立后则可列出二次方程如下:

\begin{equation}
\sqrt{\frac{nF}{lM}}\left ( \frac{Mgl}{nF}-h \right )=\frac{m\sqrt{2g\left ( h_0+h-\frac{Mgl}{nF} \right )}}{M}
\end{equation}

人的间距不能小于60cm,显然当人越多时,单人的用力大小越小,因此直接取最小间距,以使人数尽可能多:
\begin{equation}
l=\frac{l_0}{2\sin\frac{\pi}{n}}
\end{equation}
	\subsubsection{模型的求解}
通过matlab计算得出F的表达式为
\begin{equation}
\frac{5.29E4\sin\frac{\pi}{n}+2.50E7\sqrt{\left ( \frac{9.39E8\sin\frac{\pi}{n}}{7.81E12} +\frac{1.75E6\sin\frac{\pi}{n}}{3.90E11}+\frac{1.57E7}{6.25E14}\right )}+7.09E5 }{2.00E3\cdot n}
\end{equation}

并制作$Fl^{2}$关于$n$的散点图,可以看到人数小于等于20时,$Z$基本关于$n$单调递增,而规则限定$n\geq 8$,因此当队员人数为8,绳长为$0.78m$,每人用力$56.3N$,绳子低于水平面 m时,
\begin{figure}[!h]
    \centering
    \includegraphics[width=.6\textwidth]{sandian}
    \caption{散点图}
    \label{fig:sandian}
\end{figure}



%问题二

\subsection{问题二}
	\subsubsection{问题分析与建模思路}
对于第二个问题,我们忽略鼓的平动的影响,并且认为队员对球施加的力始终不变,那么,我们发现,这是一个关于鼓绕着它的质心作定点转动的过程。而一般的这种刚体运动由于它的复杂性,难以求解,因此,我们先从队员用力分布关于某一直线对称的情况讨论。因为在这种情况下,由于对称性,我们发现鼓将始终绕同一个轴进行定轴转动,而这是易于讨论并加以求解的。

	\subsubsection{模型的建立}
(一)对称模型 \par
在建模过程中,我们忽略鼓的平动的影响,并且认为队员对球施加的力为恒力。
对称模型分为如下三种:\par
(1)队员人数是偶数个,用力分布关于这些队员构成的正多边形主对角线对称,则编号如图:
\begin{figure}[!hbtp]
    \centering
    \includegraphics[width=.5\textwidth]{one}
    \caption{对称模型一}

    \label{fig:one}
\end{figure}\par
可通过单人用力的矢量与力的作用点与转轴中心的矢量叉乘,计算得出得单人力矩在y轴上分量如下:
\begin{equation}
\begin{cases}
    
\overrightarrow{OA_k}=\left ( R\cos\frac{2\pi k}{n}\cos \psi ,R\sin\frac{2\pi k}{n},R \cos\frac{2\pi k}{n}\sin\psi \right ) \\


\overrightarrow{F_k}=\left ( F_k\cos\frac{2\pi k}{n}\cos\varphi ,F_k\sin\frac{2\pi k}{n}\cos\varphi ,F_k\sin\varphi  \right ) \\

\end{cases}
\end{equation}
得到:
\begin{equation}
\left ( \overrightarrow{OA_k} \times \overrightarrow{F_k}\right )|_y=F_kR\cos\frac{2\pi k}{n}\left ( \cos\frac{2\pi k}{n}\sin\psi\cos\varphi -\cos\psi\sin\varphi  \right ) \\
\end{equation}
根据刚体定轴转动定律可列出如下方程:
\begin{equation}
A\cdot \frac{d^{2}\psi }{dt^{2} }=C\cos\psi+D\sin \psi
\end{equation}
字母的代表如下式子:
\begin{equation}
\begin{cases}
A=M\left ( \frac{R^2}{2}+\frac{d^2}{3} \right ) \\
C=2\sum_{k=1}^{\frac{n}{2}-1}F_kR\cos\frac{2\pi k}{n}\sin\varphi + F_0 R\sin\varphi- F_\frac{n}{2} R\sin\varphi
 \\ 
D=2\sum_{k=1}^{\frac{n}{2}-1}F_kR\cos^2\frac{2\pi k}{n}\cos\varphi-F_0R\cos\varphi-F_\frac{a}{2}R\cos\varphi 
\\
\end{cases}
\end{equation}

(2)队员人数是偶数个,用力分布关于这些队员构成的正多边形对边中点的连线对称,则编号如图:
\begin{figure}[!hbtp]
    \centering
    \includegraphics[width=.5\textwidth]{two}
    \caption{对称模型二}
    \label{fig:two}
\end{figure}\par

可通过单人用力的矢量与力的作用点与转轴中心的矢量叉乘,计算得出得单人力矩在y轴上分量如下:
\begin{equation}
\begin{cases}
    \overrightarrow{OA_k}=\left ( R\cos\left( \frac{2\pi k}{n}-\frac{\pi}{n}\right )\cos \psi ,R\sin\left( \frac{2\pi k}{n}-\frac{\pi}{n}\right ),R\cos\left( \frac{2\pi k}{n}-\frac{\pi}{n}\right )\sin \psi \right ) \\
    \overrightarrow{F_k}=\left ( F_k\cos\left( \frac{2\pi k}{n}-\frac{\pi}{n}\right )\cos\varphi ,F_ksin\left( \frac{2\pi k}{n}-\frac{\pi}{n}\right )\cos\varphi ,F_k\sin\varphi  \right ) \\

\end{cases}
\end{equation}
得到:
\begin{equation}
\left ( \overrightarrow{F_k} \times \overrightarrow{OA_k}\right )|_y=F_kR\cos\left( \frac{2\pi k}{n}-\frac{\pi}{n}\right )\left ( \sin\varphi\cos\psi-\cos\varphi\cos\left( \frac{2\pi k}{n}-\frac{\pi}{n}\right )\sin\psi \right )\\
\end{equation}
根据刚体定轴转动定律可列出如下方程:
\begin{equation}
A\cdot \frac{d^{2}\psi }{dt^{2} }=C\cos\psi+D\sin \psi
\end{equation}
字母的代表如下式子:
\begin{equation}
\begin{cases}
A=M\left ( \frac{R^2}{2}+\frac{d^2}{3} \right ) \\
C=2\sum_{k=1}^{\frac{n}{2}}F_kR\cos\left( \frac{2\pi k}{n}-\frac{\pi}{n}\right )\sin\varphi \\
D=-2\sum_{k=1}^{\frac{n}{2}}F_kR\cos^2\left( \frac{2\pi k}{n}-\frac{\pi}{n}\right )\cos\varphi
\\
\end{cases}
\end{equation}


(3)队员人数是奇数个,用力分布关于正多边形的某条对称轴对称,编号如图:
\begin{figure}[!hbtp]
    \centering
    \includegraphics[width=.5\textwidth]{three}
    \caption{对称模型三}
    \label{fig:three}
\end{figure}\par


可通过单人用力的矢量与力的作用点与转轴中心的矢量叉乘,计算得出得单人力矩在y轴上分量如下:
\begin{equation}
\begin{cases}
   \overrightarrow{OA_k}=\left ( R\cos\frac{2\pi k}{n}\cos \psi ,R\sin\frac{2\pi k}{n},R \cos\frac{2\pi k}{n}\sin\psi \right ) \\
   \overrightarrow{F_k}=\left ( F_k\cos\frac{2\pi k}{n}\cos\varphi ,F_k\sin\frac{2\pi k}{n}\cos\varphi ,F_k\sin\varphi  \right ) \\

\end{cases}
\end{equation}
得到:
\begin{equation}
\left ( \overrightarrow{OA_k} \times \overrightarrow{F_k}\right )|_y=F_kR\cos\frac{2\pi k}{n}\left ( \cos\frac{2\pi k}{n}\sin\psi\cos\varphi -\cos\psi\sin\varphi  \right )\\
\end{equation}
根据刚体定轴转动定律可列出如下方程:
\begin{equation}
A\cdot \frac{d^{2}\psi }{dt^{2} }=C\cos\psi+D\sin \psi
\end{equation}
字母的代表如下式子:
\begin{equation}
\begin{cases}
A=M\left ( \frac{R^2}{2}+\frac{d^2}{3} \right ) \\
C=2\sum_{k=1}^{\frac{n}{2}-1}F_kR\cos\frac{2\pi k}{n}\sin\varphi+F_kR\sin\varphi \\
D=2\sum_{k=1}^{\frac{n}{2}-1}F_kR\cos^2\frac{2\pi k}{n}\cos\varphi-F_k R\cos\varphi
\\
\end{cases}
\end{equation}

%有人先发力的模型,这里要排版一下

在有队员并不在0时刻开始用力的时候,我们假设此时所有队员的用力情况在时间中的分布仍然满足上面三种对称情况的一种。\par
我们先假设只有人提前$t_0$时间用力,由于在所有队员开始发力之前,所有的队员使用的力都是$F_0$,并且它满足

即鼓受力平衡,因此在$\left [ -t_0,0 \right ]$这段时间内,鼓的倾斜角度与时刻的关系是:

\par 而在$\left [ 0,0.1 \right ]$s时间段内,鼓的倾斜角度与时刻的关系满足所有人同时用力的情况的方程,但是由速度和位移不突变,可以得到这一段的初值条件:

\par 而当比赛队员开始用力的时刻还有更多的时候,即还有人提早了更多的时间用力或者有人推迟了用力的情况下,我们类似地也可以给出不同时间段内倾斜角与时刻的关系,并且满足相邻两段的交界时刻处角速度和倾斜角相等的边值条件,这里就不赘述了。

	\subsubsection{模型的求解}
对于题中所给的情况,我们得到结果如下:
\par 


(二)不对称模型
\subsubsection{问题分析与建模思路}
根据上面的模型以及在给出的数据中的解,我们发现,由于鼓的倾角始终很小,我们可以在允许较小的误差的前提下,认为鼓所受的力矩始终是恒定的,不受鼓的当前姿态的影响。因此,对于用力分布不对称的情况,我们可以近似地用这一简化得到下列关系:\par
(1)当没有人提前用力时,由角动量定理,我们知道  ,然后根据角加速度和角度的关系,有\par
(2)当有人提前用力,则:如果有人提前$t_0$时刻用力,则:\par
由角动量守恒,我们同样可以得到在$\left [ -t_0,0 \right ]$时间段内,角度与时间的关系,然后同样地,在$\left [ 0,0.1 \right ]$时间段内,有\par
(3)而当比赛队员开始用力的时刻还有更多的时候,即还有人提早了更多的时间用力或者有人推迟了用力的情况下,我们类似地也可以给出不同时间段内倾斜角与时刻的关系,并且满足相邻两段的交界时刻处角速度和倾斜角相等的边界条件,这里就不赘述了。
对于题目中的第8种情况,由于在-0.1s到0时间段内,我们发现它的受力情况满足对称的条件,因此,我们有\par
而在$\left [ 0,0.1 \right ]$,此时已经不再满足对称性条件,那么,我们有\par

	\subsubsection{模型的建立}


	\subsubsection{模型的求解}





 
%问题三
\subsection{问题三}
	\subsubsection{问题分析与建模思路}
由上可知在这种情况下,我们发现,如果依然按照理想情况下设计方案,那么,如果有人用力时机和用力大小有少量偏差,那么很可能会使球的上升的速度的大小和方向出现偏差,从而使其竖直分量不足以使球飞至鼓面上方40cm处,使得游戏意外结束。那么,我们需要设置一个更大的球的目标高度,以削弱这种不利的影响。

	\subsubsection{模型的建立与求解}
那么,我们在第一问的基础上增加设定的高度10cm,从而\par
(1)在第一次垫球的时候,方程 变化为\par
(2)以后的垫球中,我们将(1)中的方程中的 改变为 
求解就能得到F = 


%问题四
\subsection{问题四}
	\subsubsection{问题分析与建模思路}
从前面的结果来看,在鼓出现偏移的过程中,只要偏转角度不超过1度,我们就可以近似地认为:鼓在这一过程中做匀加速转动。

	\subsubsection{模型的建立}
因此,如图所示:我们只需要让1号位和2号位的人使用的力比稳定状态所需要的力F0各多出deltaF1和deltaF2,让 6号位和7号位的人使用的力各减小deltaF1和deltaF2即可。

由于我们可以近似地认为球与鼓发生碰撞之后,球在鼓的中心轴方向上的分速度大小不变,因此此时鼓的中心轴的倾斜方向与球的速度的倾斜方向相同,倾斜角度是球速倾斜角度的一半,即0.5度,因此10个人的合力矩的大小M满足:

又因为倾斜方向如上,故
%2*deltaF1*R*sin(fai)和2*deltaF2*R*sin(fai),M三者构成矢量三角形,故由正弦定理:


	\subsubsection{模型的求解}
通过matlab计算以上方程,我们可以得到

因此,10个人使用的力的大小应分别为      ,开始用力的时机是球将要落到鼓面上前0.1秒。
%模型的评价与推广
\section{模型的评价与推广}


%参考文献
\section{参考文献}

\begin{thebibliography}{9}%宽度9
    \bibitem[1]{liuhaiyang2013latex}
    刘海洋.
    \newblock \LaTeX {}入门\allowbreak[J].
    \newblock 电子工业出版社, 北京, 2013.
    \bibitem[2]{mathematical-modeling}
    全国大学生数学建模竞赛论文格式规范 (2019 年 9 月 12 日修改).
\end{thebibliography}







\end{document} 

%无序列表是这样的:
%\begin{itemize}
%    \item one
%    \item two
%    \item ...
%\end{itemize}
%
%有序列表是这样子的:
%\begin{enumerate}
%    \item one
%    \item two
%    \item ...
%\end{enumerate}



%下面简单介绍一下定理、证明等环境的使用。
%\begin{definition}
%    定义环境
%    \label{def:nosense}
%\end{definition}
%\cref{def:nosense}除了告诉你怎么使用这个环境以外,没有什么其它的意义。
%
%除了 definition 环境,还可以使用 theorem 、lemma、corollary、assumption、conjecture、axiom、principle、problem、example、proof、solution 这些环境,根据论文的实际需求合理使用。
%
%\begin{theorem}
%    这是一个定理。
%    \label{thm:example}
%\end{theorem}
%由\cref{thm:example}我们知道了定理环境的使用。
%
%\begin{lemma}
%    这是一个引理。
%    \label{lem:example}
%\end{lemma}
%由\cref{lem:example}我们知道了引理环境的使用。
%
%\begin{corollary}
%    这是一个推论。
%    \label{cor:example}
%\end{corollary}
%由\cref{cor:example}我们知道了推论环境的使用。
%
%\begin{assumption}
%    这是一个假设。
%    \label{asu:example}
%\end{assumption}
%由\cref{asu:example}我们知道了假设环境的使用。
%
%\begin{conjecture}
%    这是一个猜想。
%    \label{con:example}
%\end{conjecture}
%由\cref{con:example}我们知道了猜想环境的使用。
%
%\begin{axiom}
%    这是一个公理。
%    \label{axi:example}
%\end{axiom}
%由\cref{axi:example}我们知道了公理环境的使用。
%
%\begin{principle}
%    这是一个定律。
%    \label{pri:example}
%\end{principle}
%由\cref{pri:example}我们知道了定律环境的使用。
%
%\begin{problem}
%    这是一个问题。
%    \label{pro:example}
%\end{problem}
%由\cref{pro:example}我们知道了问题环境的使用。
%
%\begin{example}
%    这是一个例子。
%    \label{exa:example}
%\end{example}
%由\cref{exa:example}我们知道了例子环境的使用。
%
%\begin{proof}
%    这是一个证明。
%    \label{prf:example}
%\end{proof}
%由\cref{prf:example}我们知道了证明环境的使用。
%
%\begin{solution}
%    这是一个解。
%    \label{sol:example}
%\end{solution}
%由\cref{sol:example}我们知道了解环境的使用。
%


%
%\subsection{字体加粗与斜体}
%
%如果想强调部分内容,可以使用加粗的手段来实现。加粗字体可以用 \verb|\textbf{加粗}| 来实现。例如: \textbf{这是加粗的字体。 This is bold fonts} 。
%
%中文字体没有斜体设计,但是英文字体有。\textit{斜体 Italics}。
